\begin{longtable}{lX}
\caption{Settings for segmentation post-processing.}
\label{tab:segmentation-post-processing}\\ \hline

SegFillHoles & Removes holes in the segmented regions. \\[5pt] \hline

SegMinHoleArea & Holes with fewer pixels are filled. The default value is inf, and that means that all holes are filled. \\[5pt] \hline

SegMinArea & Regions with fewer pixels will be removed. \\[5pt] \hline

SegMaxArea & Regions with more pixels will be removed. \\[5pt] \hline

SegMinSumIntensity & Regions with a summed image intensity below this threshold will be removed. This setting will often be more useful than SegMinArea for fluorescence images. The pixel values are normalized to be between 0 and 1 and the minimum pixel value in the image is subtracted from all pixels to get rid of background illumination. \\[5pt] \hline

SegClipping & Threshold for intensity clipping, between 0 and 1. First the image is normalized so that the highest possible value is 1. Then values above the threshold are replaced by the threshold, and finally the intensities are rescaled to the original range. \\[5pt] \hline

SegWatershed & Image parameter used in a seeded watershed transform for separation of cells in clusters. \\[5pt] \hline

SegWSmooth & Standard deviation of a Gaussian smoothing kernel applied before the watershed transform. Larger values create fewer fragments and smoother separation boundaries. \\[5pt] \hline

SegWHMax & Removes local watershed minima with a depth, relative to the surrounding, below this value. Larger values create fewer fragments. \\[5pt] \hline

SegWThresh & Removes local watershed minima with an absolute depth below this value. Larger values create fewer fragments, and \setting{-inf} disables the setting. \\[5pt] \hline

SegWatershed2 & Image parameter used in a second seeded watershed transform which separates cells in clusters further. The cell outlines generated by the first watershed transform are used as input. Parameters for this transform are specified by \setting{SegWSmooth2}, \setting{SegWHMax2}, and \setting{SegWThresh2}. \\[5pt] \hline
\end{longtable} 